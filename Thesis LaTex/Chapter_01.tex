\chapter{Introduction}

In recent years, the landscape of education has experienced a significant transformation, fueled by rapid advancements in digital technology and the need to adapt to the ever-evolving demands of the 21st century   [1]    . E-learning, defined as the use of electronic technologies to facilitate the delivery and enhancement of learning experiences, has emerged as a promising solution to address the challenges faced by tertiary and higher education institutions in the pursuit of preparing students for the future [2]. This Master's Thesis aims to show the development of a new tool with state-of-the-art technologies and algorithms that will create a Video Topic Segmentation searcher for the use of students for a more efficient way of learning. 

The Thesis project is focused in the development of Topic Detection and Selection Algorithms that will look inside a database composed of educational videos that within an integration and embedding of State of the art technologies will make search inside database of videos more efficient.

The topics that are related for the development of this new technology are the following ones listed in the next table


\begin{table}[ht]
\centering
\begin{tabular}{ | m{4.5cm} | m{7 cm} | m{1.8cm} | }
\hline
\textbf{Technology} & \textbf{Explanation} & \textbf{References} \\ \hline
Speech and Transcript Generation in Videos & The architecture nowadays for doing this is with RNN and DNN that using FFTs and Puntuation Generators can give really accurate transcripts with time stamps & [3] [4] [5] [6] [7] \\ \hline
Natural Language Processing (NLP) and Natural Language Understanding (NLU) & NLP and NLU are the studies of using computational technologies and algorithms for processing, interpreting and generating text & [8] [9] [10] [11] [12] [13] \\ \hline
Large Language Models (LLM) & Large Language Models are a technology that has had an incredible quick development in recent years where Artifical Intelligence algorithms aim o interpret, understand and generate from a wide variety of topics & [14] [15] [16] [17] \\ \hline
Topic Segmentation & Models that divide and asses when a content inside of a field is related to a topic &  [18] [19] [20] [21] \\ \hline
Generative Artificial Intelligence & Generative Artificial Intelligence are technologies that use diffusion algorithms and another state-of-the-art Neural Networks and Machine Learning methodologies for generating: Texts, images, videos, etc. &  [22] [23] [24]  \\ \hline
Prompt Engineering & The new use of LLMs for helping in daily tasks have given a more important relevance to the subfield of prompt engineering: that consists in giving prompts to LLMs for generating useful results  & [25] [26] [27] \\ \hline
\end{tabular}
\caption{Overview of AI Technologies}
\label{table:ai_technologies}
\end{table}


The widespread adoption of e-learning technologies has been accelerated by numerous factors, including the increasing availability of digital devices, internet penetration, and the desire to provide cost-effective and flexible learning options. Additionally, the COVID-19 pandemic has acted as a catalyst, pushing educational institutions to adopt remote learning solutions in an effort to maintain educational continuity. [1]




Furthermore, the thesis will assess the accuracy, coherence and efiiciency of three different architectures of algorithms and technologies that will make Topi Segmentation Predictions of the Transcripts from educational videos.
In order to asses that the use of different cohenrence m

Another part that will address this Master's thesis work is that it will use State-of-the art algorithms and technologies as Chat GPT APIs that will be connected for understanding the context from  users prompts and will also generate a summary of the section from the transcript using Generative AI will give a summary and an image generated thanks to Generative Artificial Intelligence



Finally this work will show how the implementation of a searcher that can accurately make Topic Segmentation and summaries will impact the students learning. 


\label{chapter:chapter01}



\section{Problem Definition and Motivation}

Amidst the exponential progress of digital technology and the escalating demand for adaptive and inclusive learning tools in post-secondary education, the significance of e-learning platforms is progressively apparent in the educational landscape. Cutting-edge technologies, including artificial intelligence (AI) methods and tools to unveil an expand potential for refining the educational journey of students. However, integrating these sophisticated technologies into e-learning platforms constitutes a multifaceted issue, requiring a holistic comprehension of the ramifications, constraints, and optimal application strategies.

This research endeavors to address the challenge of pinpointing the use of Algorithms for creating a Topic Segmentation Searcher in a database composed of Video files that will allow students to search for specific topics inside a database made of videos. 

The motivation underpinning this research is rooted in extensive personal experiences within the sphere of education. Moreover, recent engagements as a professor in Physics and Mathematics at PrepaTec which it has provided first-hand insights into the existing educational challenges. This dual perspective as both an educator and a student has stimulated the aspiration to devise solutions and strategies addressing educational shortcomings.

In addition, this research sets out to asses not just the performance of the algorithms and the coherence, but also the impact in the efficiency of students learning outcomes.

Lastly, the goal is to construct a prototype platform that can engage with users and provide guidance on studying topics to achieve their academic objectives. For instance, users can input their academic goals, and an integrated chat bot within the platform will guide them towards suitable video content or reading materials, thereby optimizing their study time and energy in locating the necessary resources. This platform aims to offer a streamlined, efficient, and personalized educational experience for the user.





\section{Hypothesis and Research Questions}

The successful integration of cutting-edge e-learning technologies within post-secondary and higher education frameworks will manifest in heightened student engagement, motivation, and learning outcomes. Moreover, this integration should foster the creation of more inclusive, accessible, and personalized learning environments. 


Students struggle because although there are many resources available online they are not in order and there can be a lot of sources for one specific topic and a less material for another topic. The main problem is the search for cleve content that will help the student to address or review the specific topic that the student is interested.  Making an \textbf{Topic Segmentation Algorithm} that can precisely look in a database composed of videos and make an specific and useful search will help students to review and study in a more productive way.

\subsection{Research Questions:}
In light of the established hypothesis, this research intends to address the following research questions:


\begin{enumerate}
    \item  \textbf{How can an ensemble of computational algorithms and analytical methods be synergistically integrated to architect an advanced intelligent system capable of segmenting, querying, and prognosticating thematic content based on user inputs within a video database corpus?}

    \item \textbf{What methodologies can be employed to rigorously assess and quantify the performance efficiency of this newly implemented algorithmic framework in comparison to extant benchmarked systems for video content segmentation, search, and topic prediction?}

    \item \textbf{To what extent do these new pedagogical technologies enhance the efficacy and retention of learning among students? What evaluative metrics and research methodologies could be systematically applied to measure these outcomes?}

    \item \textbf{Can the integration of Generative Artificial Intelligence and Topic Segmentation techniques enhance the quality and comprehensiveness of note-taking for students?}
    
    \item \textbf{What measurable impacts do these advanced e-learning technologies have on student engagement, motivation, and learning outcomes within post-secondary and higher education contexts?} 

    
    \item \textbf{What impact does the introduction of the course content recommender have on the pace of learning and student satisfaction? How can these variables be quantitatively and qualitatively assessed? What implications does this have on the learning rate?}
    
    \item \textbf{What are the prospective avenues for the advancement and optimization of the tool developed in this thesis, and what methodological improvements could be pursued to augment its functionality and efficacy?}


    
\end{enumerate}








\section{Objectives}
In the purview of advanced e-learning technologies and their implications for post-secondary and higher education, this research seeks to achieve the following objectives:

\begin{enumerate}
    \item[\textbf{O1:}] \textbf{Conduct a comprehensive systematic literature review to elucidate the methodologies and applications of Topic Segmentation Algorithms, with the intent to inform the robust embedding and systematic design of emergent technologies in this domain.}
    \item[\textbf{O2:}] \textbf{Engineer a sophisticated algorithmic framework capable of executing precise topic segmentation searches within a video database, optimizing for both efficiency and accuracy.}
    \item[\textbf{O3:}] \textbf{Integrate Generative AI APIs and plugins within the topic-segmented sections of a video content management system to automatically generate summaries and illustrative diagrams, thereby facilitating an enhanced note-taking experience for students.}
    \item[\textbf{O4:}] \textbf{Develop a prototype platform that harnesses the capabilities of ChatGPT, integrates proprietary topic segmentation algorithms, and employs generative AI for summary generation, ensuring the platform maintains high standards of usability and user interface design.}
    \item[\textbf{O5:}] \textbf{Construct an advanced prototype of an interactive platform that leverages ChatGPT's conversational AI, amalgamates custom-developed topic segmentation algorithms, and utilizes generative AI models for the creation of concise summaries, all while upholding stringent usability protocols and aesthetic user interface principles.}
\end{enumerate}






\section{Related Work}

The scope of this research project primarily involves the examination of research questions pertaining to how advanced technologies can enhance education and promote a superior, more rapid, and highly individualized learning experience for learners. While there exists no direct research that utilizes the same technologies or methodologies proposed in this study, various papers, journals, and books related to the subject provide an instrumental foundation for the thesis.

The domains of this research project can be partitioned into four principal categories: e-learning programs and techniques, data generation from online students, implementation of state-of-the-art tools from Computer Science in education, and data analysis of student surveys and exercises.

Drawing upon these four categories, the associated works are divided as follows, providing a robust background for a structured research process with attainable and realistic objectives:

\textbf{E-learning Programs and Techniques:}
A comprehensive handbook offers a global perspective on blended learning, a convergence of online and face-to-face instruction. The handbook delves into a wide range of topics, including the design, execution, and evaluation of blended learning programs [16].

\textbf{Data Generation from Online Students:}
This paper explores learning analytics and educational data mining as budding fields aimed at enhancing student learning and improving educational systems. The authors propose guidelines for fostering effective communication and collaboration among researchers, educators, and administrators [15].

\textbf{Computer Science State-of-the-art Tools Implementations for Education:}
Various advanced technologies have been implemented or suggested for students in post-secondary and higher education with the objective of enriching future education. The following are some key studies and projects in this area:

\begin{enumerate}
    \item The report recognizes six major trends, challenges, and upcoming technologies anticipated to influence higher education in the subsequent five years, with learning analytics, adaptive learning, and artificial intelligence identified as the core technologies [17].
    \item This research proposes an interactive e-book-based flipped learning approach intended to enhance math learning effectiveness among students. The approach merges online out-of-class learning with in-class discussions and problem-solving activities [19].
    \item Knewton, an adaptive learning platform, employs artificial intelligence to tailor learning experiences for students. It assesses students' past performance, current progress, and future learning objectives to suggest the most suitable learning resources [20].
\end{enumerate}



\textbf{Data Analysis of Surveys and Exercises from Students:}
This paper introduces a novel research area named multimodal learning analytics, which amalgamates data from diverse sources such as eye-tracking, speech, and gestures to attain an in-depth understanding of students' learning processes [18].

These studies and projects provide insights into the innovative technologies and methodologies currently being developed and implemented in post-secondary and higher education. As the field continually evolves, we can anticipate further advancements in e-learning tools and platforms, shaping the future of education.









\subsection{Data Prediction Assessment}

Implementation and Feedback: The insights gained from the data analysis can be used to inform decision-making, improve educational practices, and provide personalized feedback to students. This step emphasizes the practical application of the findings to enhance students' learning experiences.


By incorporating these components into a theoretical framework, researchers and educators can systematically analyze student data to gain valuable insights and make informed decisions to enhance educational outcomes.

\subsection{Generative Language Models}

Response generation: Once the chatbot has identified the intent and gathered relevant information, it generates a response. This can be a predefined text or a dynamically generated response based on the information retrieved. Advanced chatbots use natural language generation (NLG) techniques to create human-like responses.
Machine learning and feedback loop: As users interact with chatbots, the system can learn from the user inputs and improve its understanding and response generation over time. This is achieved through a feedback loop where the chatbot's performance is evaluated, and the underlying models are updated accordingly.
User output: Finally, the chatbot's response is sent back to the user, completing the conversation loop. In the case of voice-based chatbots, the text response is converted back into speech using text-to-speech (TTS) technology.



\subsection{Education Theory}

The theoretical framework for this thesis on e-learning state-of-the-art technologies shaping the future of tertiary and higher education is grounded in a combination of educational theories and models, which provide a foundation for understanding the interplay between technology and learning. These theories and models include:

1. Constructivism: Constructivism posits that learners actively construct knowledge through their experiences and interactions with the environment . This theory provides a basis for understanding how state-of-the-art e-learning technologies can facilitate the construction of knowledge by enabling students to actively engage with learning materials, collaborate with peers, and participate in authentic learning experiences .

2. Connectivism: Connectivism emphasizes the importance of networks and connections in the learning process, and asserts that knowledge is distributed across a network of nodes. This theory is particularly relevant for understanding the potential of e-learning technologies to support the development of learning networks and communities, where learners can access, share, and create knowledge collaboratively .

3. Technology Acceptance Model (TAM): TAM is a widely used model that explains the factors influencing the adoption and use of technology. In the context of this thesis, TAM provides a framework for investigating the factors that affect the acceptance and implementation of state-of-the-art e-learning technologies in tertiary and higher education settings, such as perceived usefulness, perceived ease of use, and user attitudes.

4. Community of Inquiry (CoI) Framework: The CoI framework is a model that represents the key elements of an effective online learning environment, which include social presence, cognitive presence, and teaching presence. This framework is useful for exploring how state-of-the-art e-learning technologies can enhance the different dimensions of presence in online learning, thereby promoting a more engaging and meaningful learning experience.





