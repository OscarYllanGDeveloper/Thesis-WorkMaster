\textbf This thesis presents a comprehensive study on the utilization of existing software tools for transcribing audio from video files, coupled with an innovative approach to topic segmentation analysis. The focus lies on leveraging large language models and prompt engineering techniques to enhance the accuracy and efficiency of topic segmentation in transcribed texts. We investigate various state-of-the-art software for transcription, evaluating their performance in terms of accuracy, speed, and reliability. The transcribed texts are then subjected to a robust topic segmentation analysis, utilizing advanced natural language processing techniques and large language models. Our approach demonstrates significant improvement in identifying and categorizing topics within video content, facilitating a more nuanced and detailed content analysis. Moreover, we have developed a clean, efficient infrastructure to support our methodology, ensuring a seamless process from video input to topic segmentation. The infrastructure is designed to be scalable and adaptable, catering to the diverse needs of content analysts and researchers. Through rigorous testing and validation, our system has proven to be effective in enhancing the precision of topic searches within videos, paving the way for more accurate and insightful content analysis. This work stands as a testament to the potential of integrating existing transcription tools with cutting-edge natural language processing techniques, setting a new standard for video content analysis.\\
\textbf{Keywords}:  Natural Language Processing, Topic Modeling, Generative Language Model, Educational Data Analytics, Artificial Intelligence, E-learning Systems.


